%%%%%%%%%%%%%%%%%%%%%%%%%%%%%%%%%%%%%%%%%%%%%%%%%%%%%%%%%%%%%%%%%%%%%%%%
%    ModulateFlow Manual										       %
%                                                                      				%
%    Latex Source Code                                                  %
%                                                                     				%
%    Jorge Mariscal-Harana, adapted from P.H. Charlton           %
%%%%%%%%%%%%%%%%%%%%%%%%%%%%%%%%%%%%%%%%%%%%%%%%%%%%%%%%%%%%%%%%%%%%%%%%

%%%%% SETUP DOCUMENT %%%%%

\documentclass[12pt]{iopart}
\usepackage[pdftex]{graphicx}
\usepackage{color}
\usepackage{harvard}
\usepackage{multirow}
\usepackage{epstopdf}
\usepackage{adjustbox}
\usepackage{enumitem}
%\newcommand{\url}[1]{\textcolor{blue}{{\textit{#1}}}}
\usepackage{setspace}
\usepackage{scrextend}
\newcommand{\thresh}{\textit{k} }
\newcommand{\mean}[1]{\overline{#1}}
\newcommand{\pc}[1]{\textcolor{red}{{\textit{#1}}}}
\newcommand{\tb}[1]{\textcolor{blue}{{\textit{#1}}}}
\newcommand{\ja}[1]{\textcolor{red}{{\textit{#1}}}}
\newcommand{\eg}{\textit{e.g.} }
\newcommand{\ie}{\textit{i.e.} }
\newcommand{\mf}{\texttt{ModulateFlow}}
\newcommand{\us}{\texttt{\_}}
\usepackage{cite}
\usepackage{array} % for tables
\newcolumntype{P}[1]{>{\centering\arraybackslash}p{#1}}
\usepackage{rotating}

\makeatletter
\def\bstctlcite{\@ifnextchar[{\@bstctlcite}{\@bstctlcite[@auxout]}}
\def\@bstctlcite[#1]#2{\@bsphack
	\@for\@citeb:=#2\do{%
		\edef\@citeb{\expandafter\@firstofone\@citeb}%
		\if@filesw\immediate\write\csname #1\endcsname{\string\citation{\@citeb}}\fi}%
	\@esphack}
\makeatother

% --------------- Making bibliography -------------- %

% see: http://anorien.csc.warwick.ac.uk/mirrors/CTAN/macros/latex/contrib/IEEEtran/bibtex/IEEEtran_bst_HOWTO.pdf

% and: http://tex.stackexchange.com/questions/164017/limiting-the-number-of-authors-in-the-references-with-ieeetran
\usepackage{filecontents} % To make the bib-file

\begin{filecontents}{refs.bib}
	@IEEEtranBSTCTL{IEEEexample:BSTcontrol,
		CTLuse_url = "no",
		CTLuse_forced_etal       = "yes",
		CTLmax_names_forced_etal = "3",
		CTLnames_show_etal       = "1", 
		CTLdash_repeated_names = "no" 
	}
	@article{paperOne,
		author = "Author First and Author Second and Author Third and Author Fourth",
		title = "Paper One Title",
		journal = "Awesome Journal",
		pages = "111--115",
		year = 2013
	}
	@incollection{paperTwo,
		author = "Author First and Author Second and Author Third",
		title = "Paper Two Title",
		booktitle = "Proc. of Collection",
		pages = "222--225",
		year = 2013
	}
\end{filecontents}



\begin{document}

\bstctlcite{IEEEexample:BSTcontrol}

%%%%% TITLE AND AUTHORS %%%%%

\title[J Mariscal-Harana \etal]{\mf{} \\ Modulating diastolic flow or velocity waveforms}

\author{Jorge Mariscal-Harana} % $^{1,2}$

\address{Department of Biomedical Engineering, King's College London, UK}

% \ead{peter.charlton@kcl.ac.uk}

%%%%% VERSION INFORMATION %%%%%

%\vspace{10pt}
%\begin{indented}
%\item[]Draft from \today
%\end{indented}

%%%%% ABSTRACT %%%%%

\begin{abstract}

This report presents \mf, a script for modulating diastolic flow or velocity waveforms.

\end{abstract}

%%%%% KEYWORDS %%%%%

% \noindent{\it Keywords}: respiratory modulation, biomedical signal processing, electrocardiography, photoplethysmography, respiratory rate

%%%%% SUBMISSION TO JOURNAL %%%%%
% \submitto{\PM}

%%%%% SPACING %%%%%
%\doublespacing

%%%%% INTRODUCTION %%%%%

\section{Summary}
\label{sec:intro}

Clinical blood flow or velocity waveform pre-processing can be difficult due to a low sampling rate (SR) and/or noise. 
This report describes how \mf{} deals with these two common problems by (a) using a cubic spline interpolation for low SRs; and (b) modulating the flow or velocity waveform during diastole to eliminate noise after valve closure.

\section{Methods}

\mf{} consists of three stages: (i) waveform interpolation; (ii) noise thresholding; and (iii) downslope projection. The methods used at each stage are now described in turn.

\subsection{Waveform interpolation}

If the number of waveform datapoints is smaller than 100, a cubic spline interpolation (\texttt{spline}) is applied to increase the number of datapoints to 100. 

\subsection{Noise thresholding}

The following steps are followed to threshold noise (e.g. US speckle) during diastole:
\begin{itemize}
	\item The maximum flow or velocity datapoint is identified. Previous datapoints (systolic upstroke) are excluded from the analysis.
	\item Datapoints whose value is < 20$%$ of the maximum value are identified. Values after the first datapoint to satisfy this condition are set to 0.
	
\end{itemize}

\subsection{Downslope projection}







\section{Using \mf}

\mf{} can be used to calculate CV indices from either a pulsatile signal containing several pulses, or a single pulse wave. The input data, $S$, should be prepared as a structure with two fields: $S.v$, a vector of signal amplitudes, and $S.fs$, the sampling frequency of the signal. At its simplest, \pa can be called using
\begin{center}
	\texttt{cv\us inds = PulseAnalyse(S);}
\end{center}
where \texttt{cv\us inds} is a structure containing individual fields for each of the calculated CV indices (named according to the abbreviations listed in Table \ref{tab:SIs_res}). Each index's field is itself a structure, containing the calculated value (\eg{} \texttt{cv\us inds.SI.v}, which is a median in the case of multiple pulses), and raw values for each pulse (\texttt{cv\us inds.SI.raw}) if the input signal contains multiple pulses.

Additional functionality can be exploited by specifying additional inputs. Firstly, if the subject's height is provided in $S.ht$ then those CV indices which require height will be calculated. Secondly, options can be specified as a second input using
\begin{center}
	\texttt{cv\us inds = PulseAnalyse(S, options);}
\end{center}
where \texttt{options} is a structure containing the following logical fields:
\begin{itemize}
	\item \texttt{exclude\us low\us quality\us data}: whether or not to exclude pulses with a low signal quality in the calculation of a median value for each CV index (only applicable when using an input signal with multiple pulses).
	\item \texttt{do\us plot}: whether or not to plot an example pulse with fiducial points annotated (similar to those in Figure \ref{fig:fid_pts}).
\end{itemize}

Additional outputs can also be obtained using
\begin{center}
	\texttt{[cv\us inds, fid\us pts, pulses, S\us filt] = PulseAnalyse(S);}
\end{center}
where \texttt{fid\us pts} is a structure containing the indices of each fiducial point; \texttt{pulses} is a structure containing the indices of the onsets and peaks of each pulse, and the signal quality of each pulse (a logical where 1 indicates high signal quality); and \texttt{S\us filt} is the filtered pulsatile signal to which these indices correspond.

\section{Further Work}
\label{sec:further_work}

Ideas for improvement.

\newpage


\section*{References}


\bibliographystyle{IEEEtran}
\bibliography{refs,refs2,refs_extra}

\end{document}

